\documentclass[notitlepage]{report}

\input{~/Physics/Latex/preamble}
\input{~/Physics/Latex/macros}
\input{~/Physics/Latex/letterfonts}

\title{\Huge{Electrolitic}\\Exercise 3}
\author{Alon Ner Gaon}

\begin{document}

\maketitle
\[
	F^{\mu\nu}=
	\begin{pmatrix}
	 0 & E_x & E_y & E_z\\
	 -E_x & 0 & -B_z & B_y\\
	 -E_y & B_z & 0 & -B_x\\
	 -E_z & -B_y & B_x & 0
	\end{pmatrix}
.\] 
\begin{question}[title = Question 1.1]{}{}
Show that from the equation:\\
\[
	m\frac{\text{d}U^{\mu}}{\text{d}\tau}=\frac{q}{c}F^{\mu\nu}U_\nu
.\] 
The lorentz force and the work eq. can be derived.\\
Which of the fields $\vec{E},\ \vec{B}$ execute work?
\end{question}
\textbf{\emph{\underline{Solution}:}}\\
For $\mu=0$ :\\
\[
	m \frac{\text{d}U^{0}}{\text{d}\tau} = \frac{q}{c}\left(\vec{E}\cdot \vec{v}\right)=\frac{1}{c}\left(\vec{f}\cdot \vec{v}\right)=\frac{1}{c}\frac{\text{d}E}{\text{d}t}  
.\] 
For $\mu=i$ :\\
\[
	m \frac{\text{d}U^{i}}{\text{d}\tau}  =\frac{q}{c}\left(F^{i0}U_0+F^{0i}U_i\right)  = \frac{q}{c}\gamma\left(c \vec{E}+\vec{B}\times \vec{v}\right)  
.\] 
Only the electric field execute work because the term $\vec{B}\times \vec{v}$ is perpendicular to $\vec{v}$ and will be nullified in the dot product $\vec{f} \cdot \vec{v}$.
\begin{question}[title = Question 1.2]{}{}
Find $\vec{r}\left(t\right) $ of a relativistic particle in a constant magnetic field, and show the non-relativistic approximation. What's the diffrence between the case of a constant electric field and the case of a magnetic one?
\end{question}
\textbf{\emph{\underline{Solution}:}}\\
We'll use our freedom of choise of a coordinate system and pick $\vec{B}$ such that:
\begin{align*}
	\vec{B} =&\ B\hat{z}\\
	\vec{v}\left(t=0\right) =&\ v_{0x}\hat{x}+v_{0y}\hat{y} 
.\end{align*}
\end{document}
