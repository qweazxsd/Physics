\documentclass[notitlepage]{report}

\input{~/Physics/Latex/preamble}
\input{~/Physics/Latex/macros}
\input{~/Physics/Latex/letterfonts}

\title{\Huge{Solid State}\\Exercise 3}
\author{Alon Ner Gaon}

\begin{document}

\maketitle

\begin{question}[title = Question 1]{}{}
Show that:\\
\[
	R_H= \frac{1}{ce} \frac{n_h\mu_h^2-n_e\mu_e^2}{\left(n_h\mu_h+n_e\mu_e\right)^2 }
.\] 
\tcblower
\textbf{\emph{\underline{Solution}:}}\\
Firstly we'll note that there's no net flow in $\hat{y}$:\\
\[
	J_y=n_eq_ev_{ye}+n_hq_hv_{yh}=0
.\] 
For every charge carier:\\
	\[
	0=-\frac{\vec{p}}{\tau}+q\left[\vec{E}+\frac{1}{c}\vec{v}\times \vec{H}\right] 
	.\] 
The equation for $\hat{x}$ :\\
\begin{equation}
	mv_x=\tau qE_x+q\frac{\tau}{c}v_yH\label{eq1}
.\end{equation}
And the equation for $\hat{y}$ :\\
\begin{equation}
	mv_y=\tau qE_y-q\frac{\tau}{c}v_xH\label{eq2}
.\end{equation}
We'll take $v_x$ from \eqref{eq1} and place in \eqref{eq2}\\
\begin{gather*}
	mv_y = \tau qE_y-q\frac{\tau}{c}\left(\frac{\tau q}{m}E_x+q\frac{\tau}{cm}v_yH\right)H\label{eq2}\\
mv_y\left(1+\left(\frac{\tau qH}{cm}\right)^2 \right) = \tau qE_y-\frac{\tau^2q^2H}{cm}E_x
\end{gather*}
And if $\omega_c\tau\equiv \frac{qH\tau}{cm}\ll1$ we'll throw $\omega_c^2$ and leave $\omega_c^1$:\\
\[
	v_y=\underbrace{\frac{\tau q}{m}}_{\mu}\left(E_y\pm\omega_c\tau E_x\right) 
.\] 
We'll note that $\omega_c$ is positive or negative depending on the sign of $q$.\\
\tcbbreak
Now we'll put $v_y$ for every charge carier in the expression for $J_y$:\\
\begin{gather*}
	J_y=n_eq_e\mu_e\left(E_y-\omega_c\tau_eE_x\right)+n_hq_h\mu_h\left(E_y+\omega_c\tau_h E_x\right)=0\\
\Rightarrow E_ye\left(n_e\mu_e+n_h\mu_h\right)=\frac{He}{c}E_x\left(n_h\mu_h^2-n_e\mu_e^2\right)  
.\end{gather*}
Again, if $\omega_c\tau\ll 1$:\\
\[
	v_x=\pm\mu E_x
.\] 
And:\\
\begin{gather*}
	J_x=n_eq_ev_{xe}+n_hq_hv_{xh}\\
\Rightarrow E_x=\frac{1}{e\left(n_h\mu_h+n_e\mu_e\right) }J_x	
\end{gather*}
So finally:\\
\begin{gather*}
E_ye\left(n_e\mu_e+n_h\mu_h\right)=\frac{He}{c}\frac{1}{e\left(n_h\mu_h+n_e\mu_e\right)}J_x\left(n_h\mu_h^2-n_e\mu_e^2\right)\\
\Rightarrow R_H\equiv \frac{E_y}{HJ_x}=\frac{1}{ce} \frac{n_h\mu_h^2-n_e\mu_e^2}{\left(n_h\mu_h+n_e\mu_e\right)^2 }
\end{gather*}
\end{question}
\begin{question}[title = Question 2]{}{}
\textbf{\emph{\underline{Solution}:}}\\
\[
	n=\frac{1}{23}\frac{mol}{gr}\cdot 1 \frac{gr}{cm^3}=\frac{N_a}{23}\frac{particles}{cm^3}=\frac{N_a\cdot10^{6}}{23} \frac{1}{m^3}
.\] 
\begin{align*}
	\left[J\right] =&\ \frac{qm}{m^3s} \\
	\left[I\right] =&\ \frac{q}{s} \\
	\Rightarrow I =&\ AJ
\end{align*}
\begin{gather*}
E_y = R_HHJ_x\\
\Rightarrow \underbrace{A}_{L^2}E_y = R_HHI\\
\underbrace{LE_y}_{V_y} = \frac{R_HHI}{L}=\frac{23}{5eN_a\cdot10^{3}}\approx \boxed{-4.7675\cdot 10 ^{-8}\quad V}
.\end{gather*}
\end{question}
\newpage
\begin{question}[title = Question 3]{}{}
\textbf{\emph{\underline{Solution}:}}\\
The two surfaces genarate a field in the same diraction, so the total field is $E=4\pi\sigma$.\\
\begin{gather*}
	F=qE\\
	m\ddot{d}=-e 4\pi\sigma=-e 4\pi ned\\
	\ddot{d}=- \underbrace{\frac{4\pi ne^2}{m}}_{\omega_p^2}d
\end{gather*}
Which means $d$ oscillates with angular velocity $\omega_p$.
\end{question}
\begin{question}[title = Question 4.1]{}{}
\textbf{\emph{\underline{Solution}:}}\\
We'll write the equation of motion for the electron:\\
\[
	0=-\frac{\vec{p}}{\tau}-e\frac{1}{c}\vec{v}\times \vec{H} 
.\] 
After a short time there's no magnetic field, so in $\hat{y}$:\\
\[
	v_y=\omega_c t v_x
.\] 

\end{question}
\begin{question}[title = Question 4.2 + 4.3]{}{}
\textbf{\emph{\underline{Solution}:}}\\
The flow of energy at $x$ in the $\hat{y}$ direction due to the magnetic field:\\
\begin{align*}
	J_{Q\ y}^{\left(magnetic\right) } =&\ \frac{1}{2}nv_y\left[\epsilon\left(x-v_x\tau\right)-\epsilon\left(x+v_x\tau\right)  \right]\\
	=&\ -nv_yv_x\tau\frac{\partial\epsilon}{\partial x}\\
	=&\ -n\omega_cv_x^2\tau^2\frac{\partial\epsilon}{\partial x}\\
	=&\ -\frac{1}{3}n\omega_cv^2\tau^2\frac{\partial\epsilon}{\partial T}\frac{\partial T}{\partial x} \\
	=&\ -\frac{1}{3}\omega_cv^2\tau^2c_v\frac{\partial T}{\partial x} \\
\end{align*}
In the same fashion, the flow of energy due to the temperature gradient:\\
\begin{align*}
	J_{Q\ y}^{\left(\nabla T\right) } =&\ \frac{1}{2}nv_y\left[\epsilon\left(y-v_y\tau\right)-\epsilon\left(y+v_y\tau\right)  \right]\\
	=&\ -nv_y^2\tau\frac{\partial\epsilon}{\partial y}\\
	=&\ -\frac{1}{3}nv^2\tau\frac{\partial\epsilon}{\partial T}\frac{\partial T}{\partial y} \\
	=&\ -\frac{1}{3}v^2\tau c_v\frac{\partial T}{\partial y}\equiv-\kappa\frac{\partial T}{\partial y}
.\end{align*}
But the two flows must nullify each other for the conductor is finite in $\hat{y}$.\\
\begin{gather*}
	J_{Q\ y}^{\left(\nabla T\right) } = -J_{Q\ y}^{\left(magnetic\right) }\\
-\cancel{\frac{1}{3}}\cancel{v^2}\tau \cancel{c_v}\frac{\partial T}{\partial y} = \cancel{\frac{1}{3}}\omega_c\cancel{v^2}\tau^{\cancel{2}} \cancel{c_v}\frac{\partial T}{\partial x} \\
\Rightarrow \boxed{\frac{\partial T}{\partial y} = -\omega_c\tau\frac{\partial T}{\partial x}
}.\end{gather*}
\end{question}
\end{document}
